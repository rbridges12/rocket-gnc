\documentclass[11pt]{article}
\usepackage{graphicx}
\usepackage{amsmath, amssymb, float}
\usepackage{mathtools}
\usepackage{enumitem}
\allowdisplaybreaks

\DeclareMathOperator{\EX}{\mathbb{E}}% expected value
\DeclareMathOperator{\Var}{\operatorname{Var}}% variance
\DeclareMathOperator{\Span}{\operatorname{span}}% span
\DeclareMathOperator{\Ad}{\operatorname{Ad}}% Adjoint
\newcommand{\pd}[2]{\frac{\partial #1}{\partial #2}} % partial derivatives

\title{Notes}
\author{rlybrdgs }

\begin{document}

\maketitle
\section{3D Dynamics}
3D Dynamics with fixed mass, using a rotation matrix to represent orientation:
\begin{align*}
    \mathbf{x} &= \begin{bmatrix}
        \mathbf{p} \\ \mathbf{R} \\ \mathbf{v} \\ \mathbf{\omega}
    \end{bmatrix}, \quad
    \mathbf{u} = \begin{bmatrix}
        \tau_x \\ \tau_y \\ \tau_z
    \end{bmatrix} \\
    \dot{\mathbf{p}} &= \mathbf{v} \\
    \dot{\mathbf{R}} &= \mathbf{R} \hat{\omega} \\
    \dot{\mathbf{v}} &= \frac{1}{m} \mathbf{R} \begin{bmatrix}
        f_T - f_D \\ 0 \\ 0
    \end{bmatrix} + \begin{bmatrix}
        0 \\ 0 \\ -g
    \end{bmatrix} \\
    \dot{\mathbf{\omega}} &= \mathcal{I}^{-1} \left( \mathbf{u} - \mathbf{\omega} \times \mathcal{I} \mathbf{\omega} \right)
\end{align*}
Where $\hat{\omega}$ is the skew-symmetric matrix of $\mathbf{\omega}$, $f_T$ is the thrust force, $f_D$ is the drag force, $m$ is the mass, $g$ is the acceleration due to gravity, and $\mathcal{I}$ is the inertia matrix. We can combine these equations to get the full dynamics:
\begin{align*}
    \dot{\mathbf{x}} &= f(\mathbf{x}, \mathbf{u}, t) \\
    \begin{bmatrix}
        \dot{\mathbf{p}} \\
        \dot{\mathbf{R}} \\
        \dot{\mathbf{v}} \\
        \dot{\mathbf{\omega}}
    \end{bmatrix}
     &= \begin{bmatrix}
        \mathbf{v} \\
        \mathbf{R} \hat{\omega} \\
        \frac{f_T(t) - f_D(\mathbf{x})}{m} \mathbf{R} \mathbf{e}_1 - g\mathbf{e}_3 \\
        \mathcal{I}^{-1} \left( \mathbf{u} - \mathbf{\omega} \times \mathcal{I} \mathbf{\omega} \right)
     \end{bmatrix} \\
     f_D(\mathbf{x}) &= \frac{1}{2} \rho \|\mathbf{v}\|^2 C_D A
\end{align*}

\end{document}